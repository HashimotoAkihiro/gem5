\hypertarget{NetworkInterface__d_8cc}{
\section{mem/ruby/network/garnet/fixed-\/pipeline/NetworkInterface\_\-d.cc}
\label{NetworkInterface__d_8cc}\index{mem/ruby/network/garnet/fixed-\/pipeline/NetworkInterface\_\-d.cc@{mem/ruby/network/garnet/fixed-\/pipeline/NetworkInterface\_\-d.cc}}
}
{\ttfamily \#include $<$cassert$>$}\par
{\ttfamily \#include $<$cmath$>$}\par
{\ttfamily \#include \char`\"{}base/cast.hh\char`\"{}}\par
{\ttfamily \#include \char`\"{}base/stl\_\-helpers.hh\char`\"{}}\par
{\ttfamily \#include \char`\"{}debug/RubyNetwork.hh\char`\"{}}\par
{\ttfamily \#include \char`\"{}mem/ruby/buffers/MessageBuffer.hh\char`\"{}}\par
{\ttfamily \#include \char`\"{}mem/ruby/network/garnet/fixed-\/pipeline/NetworkInterface\_\-d.hh\char`\"{}}\par
{\ttfamily \#include \char`\"{}mem/ruby/network/garnet/fixed-\/pipeline/flitBuffer\_\-d.hh\char`\"{}}\par
{\ttfamily \#include \char`\"{}mem/ruby/slicc\_\-interface/NetworkMessage.hh\char`\"{}}\par
